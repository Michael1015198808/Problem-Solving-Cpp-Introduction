\documentclass{article}
\usepackage{xcolor}
\usepackage{listings}
\definecolor{codegreen}{rgb}{0,0.6,0}
\definecolor{codegray}{rgb}{0.5,0.5,0.5}
\definecolor{codepurple}{rgb}{0.58,0,0.82}
\definecolor{backcolour}{rgb}{0.95,0.95,0.92}
\lstdefinestyle{mystyle}{
    backgroundcolor=\color{backcolour},   
    commentstyle=\color{codegreen},
    keywordstyle=\color{magenta},
    numberstyle=\tiny\color{codegray},
    stringstyle=\color{codepurple},
    basicstyle=\footnotesize,
    breakatwhitespace=false,         
    breaklines=true,                 
    captionpos=b,                    
    keepspaces=true,                 
    numbers=left,                    
    numbersep=5pt,                  
    showspaces=false,                
    showstringspaces=false,
    showtabs=false,                  
    tabsize=2
}
\lstset{style=mystyle}
\title{Exmplanation of some C++ feature grammers}
\author{Zhenyu Yan\\TA of Problem Solving I}
\date{}

\begin{document}
\maketitle
Pre-requisites: None.
\section{std::cin and std::cout}
\begin{lstlisting}[language = C++]
    #include <iostream>
    //This is a header for C++. Which means IO(input/output) stream
    // It includes cin, cout, cerr, clog, and many other things.
    // What we care most here is just cin and cout

    #include <stdio.h>
    // It's legal to include any C header in C++!
    //Some time, people consider using "c***" instead of "***.h"
    //For example, #include <stdio.h> can be replaced by #include <cstdio>
    //#include <math.h> can be replaced by #include <cmath>
    //#include <ctype.h> can be replaced by #include <cctype>
    //Notice that string.h is the same(Not actually) as cstring, but not string!
    //#include <string> will allow you to use a thing called std::string.
    //Which will be covered in further OJ.(maybe in two weeks or three)

    int main() {
        int n;
        char s[100];
        std::cin >> n;
        //std::cin is a tool for C++ to read input. It's quite simple to use it,
        //(But it's sometimes much too slower)
        //you just std::cin >> what_you_want_to_read;
        //What's coolest is that you don't need to care about the type
        //std::cin uses some things called
        //operator overload and function overload to implements it.
        //Which will also be cover latter.
        std::cin >> s;
        //Input a string is similar
        std::cout << "The number you input is ";
        //std::cout uses the same technique
        std::cout << n;
        std::cout << "\n";
        std::cout << "The string you input is " << s << "\n";
        //std::cout also allows you to output in this way
        //you can think that
        //(std::cout << "The string you input is ")
        //is still cout, so you can continue on << s
    }
\end{lstlisting}
(Check out cin\_cout.cpp)\\
sample input
\begin{lstlisting}
100
Hello
\end{lstlisting}
output
\begin{lstlisting}
The number you input is 100
The string you input is Hello
\end{lstlisting}
\section{namespace}
You may see a lot of things like std:: above. What does it mean?\\
It's a namespace. A simple way to understand is that, C++ provides a lot of tools. But if they come to play once you include them, things may be awful somethings. Because you may mistakely call something the library provides.\\
So there is something called namespace. So that you can use the name cin, cout for some other useges.\\
Well, to make the code beautiful, there is also some ways to make ugly std::cin into beautiful cin\\
That is
\begin{lstlisting}
    using std::cin;
\end{lstlisting}
Which allows you to use cin as a shorthand for std::cin\\
Sometimes, you can also use
\begin{lstlisting}
    using namespace std;
\end{lstlisting}
Which allows you to use all things without std:: !\\
(Check out namespace\_1.cpp and namespace\_2.cpp)
\newpage
\end{document}

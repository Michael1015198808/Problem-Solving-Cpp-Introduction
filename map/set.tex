\documentclass{article}
\usepackage{xcolor}
\usepackage{listings}
\definecolor{codegreen}{rgb}{0,0.6,0}
\definecolor{codegray}{rgb}{0.5,0.5,0.5}
\definecolor{codepurple}{rgb}{0.58,0,0.82}
\definecolor{backcolour}{rgb}{0.95,0.95,0.92}
\lstdefinestyle{mystyle}{
    backgroundcolor=\color{backcolour},   
    commentstyle=\color{codegreen},
    keywordstyle=\color{magenta},
    numberstyle=\tiny\color{codegray},
    stringstyle=\color{codepurple},
    basicstyle=\footnotesize,
    breakatwhitespace=false,         
    breaklines=true,                 
    captionpos=b,                    
    keepspaces=true,                 
    numbers=left,                    
    numbersep=5pt,                  
    showspaces=false,                
    showstringspaces=false,
    showtabs=false,                  
    tabsize=2
}
\lstset{style=mystyle}
\title{Map}
\author{Zhenyu Yan\\TA of Problem Solving I}
\date{}

\begin{document}
\maketitle
Pre-requisites: Cpp basic features.
\section{What is std::set}
std::map(hereinafter called map) can be regarded as a mapping from variables of one type to another type.\\
You can also treat it as an "array"(Though it's actually much slower than arrays and vectors)
\section{What's the usage of map}
As I said, we can treat map as a mapping. So we can use it to maintain a kind of relationship.\\
e.g., if we would like to have a mapping from students to their scores, we can write(Where Student and Score are two classes we define)\\
\begin{lstlisting}
map<Student, Score>score_list;
\end{lstlisting}
Wait, why we write two types between angle brackets?\\
As I said, we would like to have a mapping from one type to another. It's much cooler than a mapping that just map an int into another int, a double into another double. Right?\\
As I said, we can also treat map as an array.
 e.g., if we would like to store information of students. It's impossible that we declare an array like
\begin{lstlisting}
Student[191999999]students;
\end{lstlisting}
Becausee this is a huge waste of memory!\\
But we can write
\begin{lstlisting}
map<int, Student>
\end{lstlisting}
(Detailed explanation will be attached below)
\newpage
\section{Usage of map}
\begin{lstlisting}
//points.cpp
#include <iostream>
#include <map>

using namespace std;
struct Point {
    int x, y;
};

int main() {
    map<int, Point> points;
    for(int i = 0; i < 10; ++i) {
        Point p;
        p.x = i;
        p.y = 10 - i;
        points[i] = p;
    }
    for(int i = 0; i < 10; ++i) {
        cout << points[i].x << ',' << points[i].y << "\n";
    }
    return 0;
}
\end{lstlisting}
desired output
\begin{lstlisting}
0,10
1,9
2,8
3,7
4,6
5,5
6,4
7,3
8,2
9,1
\end{lstlisting}
\newpage
\begin{lstlisting}
//fake_array.cpp
#include <iostream>
#include <map>

using namespace std;

int main() {
    map<int, int> fake_array;
    fake_array[-1] = -999;
    fake_array[1] = 999;
    fake_array[10000] = 1;
    fake_array[20000] = 4;
    fake_array[10000000] = 9;
    int i;
    while(cin >> i) {
        if(fake_array.count(i)) {
            cout << fake_array[i] << '\n';
        } else {
            cout << "No such element in this \"array\"\n";
        }
    }
    return 0;
}
\end{lstlisting}
You can input any number. If you number is some indexes we have assigned, the element will be printed.\\
So, as you can see
\begin{lstlisting}
map<int, another_type>
\end{lstlisting}
is just like a huge "array", which saves a lot of memory for storaing discrete indexes.
\end{document}

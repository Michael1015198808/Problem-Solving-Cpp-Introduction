\documentclass{article}
\usepackage{xcolor}
\usepackage{listings}
\definecolor{codegreen}{rgb}{0,0.6,0}
\definecolor{codegray}{rgb}{0.5,0.5,0.5}
\definecolor{codepurple}{rgb}{0.58,0,0.82}
\definecolor{backcolour}{rgb}{0.95,0.95,0.92}
\lstdefinestyle{mystyle}{
    backgroundcolor=\color{backcolour},   
    commentstyle=\color{codegreen},
    keywordstyle=\color{magenta},
    numberstyle=\tiny\color{codegray},
    stringstyle=\color{codepurple},
    basicstyle=\footnotesize,
    breakatwhitespace=false,         
    breaklines=true,                 
    captionpos=b,                    
    keepspaces=true,                 
    numbers=left,                    
    numbersep=5pt,                  
    showspaces=false,                
    showstringspaces=false,
    showtabs=false,                  
    tabsize=2
}
\lstset{style=mystyle}
\title{Vector}
\author{Zhenyu Yan\\TA of Problem Solving I}
\date{}

\begin{document}
\maketitle
Pre-requisites: Cpp basic features.
\section{What is std::vector}
std::vector(hereinafter called vector) can be regarded as a dynamic-size array.\\
\section{What's the usage of vector}
The most common usage of vector is array of unknown size.\\
For example, in OJ1-3A food chain. We encountered the problem that, if a creature eats many other creatures, it'll be hard to record this.\\
If you declare a \begin{lstlisting}
int eat[10000][10000];\end{lstlisting}
It'll be a greate cost of memory. This is where vector comes to play.\\
It automatically controls the memory it uses. So\begin{lstlisting}
vector<int> eat[10000];\end{lstlisting}
will be a wonderful choice.
\newpage
\section{Usage of vector}
\begin{lstlisting}
#include <iostream>
#include <vector>
//In C++, things you wanna use will usually be in the header with same name!
//introducing std::string by #include <string>
//introducing std::queue by #include <queue>

using namespace std;

int main() {
    int n;
    cin >> n;
    vector<int> int_vector;
    //vector<int> here means that you want a vector, that use to stroe int.
    //vector<double> will give you a vector to store double
    //So, what will vector<vector<int> > give you?
    for(int i = 0; i < n; ++i) {
        int_vector.push_back(i);
        //What does the dot here means????
        //In C++, struct and class are allow to have member functions
        //which can be regarded as "function" asscotiated to that struct/class instantce.
        //You can regard it as something similar to
        //push_back(int_vector, i);
        //or
        //push_back(&int_vector, i);
        //(Use pointer because the vector may be modified in that function)
        //See next section for more examples
    }
    for(int i = 0; i < n; ++i) {
        cout << int_vector[i] << "\n";
        //Here, you can use int_vector like an array!
        //It's so coooooooool, isn't it?
        //But, if your index is out of bound, it booms right away.
        //(Arrays allows a little out of bound most of times, that's why so many students in our class passed OJ with wrong code)
    }
    return 0;
}

\end{lstlisting}
(Check out vector.cpp)\\
sample input
\begin{lstlisting}
5
\end{lstlisting}
sample output
\begin{lstlisting}
0
1
2
3
4
\end{lstlisting}
\end{document}
